% !TeX encoding = UTF-8
% !TeX spellcheck = bg_BG_IV
\documentclass{article}
\usepackage[X2]{fontenc}
\usepackage{newunicodechar}
%\DeclareUnicodeCharacter{0466}{\CYRLYUS}
%\DeclareUnicodeCharacter{0467}{\cyrlyus}

% iotated characters
%\DeclareUnicodeCharacter{046C}{\CYRII\CYRBYUS}
%\DeclareUnicodeCharacter{046D}{\cyrii\cyrbyus}



\usepackage[bulgarian]{babel}
%\usepackage{cmsrb}

%\newcommand{\YUSligature}{s}
%\newunicodechar{ѭ}{\YUSligature}
\newunicodechar{й}{i}
\newunicodechar{і}{\cyrii}
%opening
\title{Заправилъ Господъ манастиръ}
\author{народенъ}

\begin{document}

\maketitle

Заправилъ Господъ манастиръ \\
само отъ душы праведны, \\
една му душа не стига 
манастиръ да си доправи \\

Че извика Господъ ангела, \\
ангела -- светий архангелъ. \\

Че го на земля изпрати \\
праведна душѫ да найдетъ. \\

Ходилъ е ангелъ, търсилъ е \\
нигдѣ душѫ не найде. \\

Въ едно ми село накрая, въ една кѫща вдовишка  \\
праведна душа намѣри. \\
Вдовица лежи, умира, деца край нея пищѣха: \\

\textbf{Като си, мамо, умирашъ, назы на кого оставяшъ? }\\

Сѣдѣлъ е ангелъ, слушалъ е.  \\
Много се жалба нажали,  \\
право при Господъ отиде и Господу думаше: \\

Не можахъ, Боже, не можахъ душѫ вдовишка да взематъ. \\
Децата жално плачеха, плачеха и се молѣха. \\
Не можахъ душѫ да взематъ, седемь сирака оставитъ. \\
 
 
% На гроба ми изникна-щѫтъ цвѣтя –
% това са мойтѣ пѣсни недопѣти.
% А между тѣхъ изникна-ще и тя –
% от хубави най-хубавото цвѣте 
 
%Най-хубавото цвѣте не възпѣхъ!
%С вълшебенъ дъхъ душѫ ми възхитена
%то упои... и в нея съ трепетъ плахъ
%увѣхна мойта пѣсень неродена!


\end{document}
