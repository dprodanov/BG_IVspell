% !TeX spellcheck = bg_BG_IV
% !TeX encoding = UTF-8

\documentclass{article}

\usepackage[utf8]{inputenc}
\usepackage[X2]{fontenc}
\DeclareUnicodeCharacter{0466}{\CYRLYUS}
%\DeclareTextCommand{\CYRLYUS}{X2}{Ѧ} 
\DeclareUnicodeCharacter{0467}{\cyrlyus}
\DeclareTextCommand{\cyrlyus}{X2}{ѧ} 
%\usepackage{ucs}
%\usepackage{autofe}
\usepackage{newunicodechar}

%\tracinglostchars=3 

%\makeatletter
%\@tfor\next:=%
%{А}{а}{Б}{б}{В}{в}{Г}{г}{Д}{д}{Е}{е}
%{Ж}{ж}{З}{з}{И}{и}{Й}{й}{К}{к}
%{Л}{л}{М}{м}{Н}{н}{О}{о}{П}{п}{Р}{р}
%{С}{с}{Т}{т}{У}{у}{Ф}{ф}{Х}{х}{Ц}{ц}
%{Ч}{ч}{Ш}{ш}{Щ}{щ}{Ъ}{ъ}{Ь}{ь}
%{Ю}{ю}{Я}{я}{Ѣ}{ѣ}{Ѫ}{ѫ}{Ѧ}{ѧ}%
%\do{%
%	\begingroup\def\IeC{\protect\DeclareTextSymbolDefault}%
%	\protected@edef\@temp{\endgroup\csname u8:\next\endcsname{X2}}\@temp}
%\makeatother



% iotated characters
\newunicodechar{Ѭ}{\CYRII\CYRBYUS}
\newunicodechar{ѭ}{\cyrii\cyrbyus}
\newunicodechar{Ꙑ}{\CYRII\CYRHRDSN}
\newunicodechar{ꙑ}{\cyrii\cyrhrdsn}

%\newunicodechar{ѧ}{\cyrlyus}

\usepackage[bulgarian]{babel}

\begin{document}
	\title{Иванчевски правописъ}
\maketitle

Речникъ на българския езикъ, съдържащъ слова и тѣхнитѣ форми споредъ изискванията на Иванчевския правописъ. 
Иванчевски правописъ е името, давано понѣкога на първиятъ официаленъ български правописъ, 
въвѣденъ съ наредба на министъра на народното просвѣщение Тодоръ Иванчовъ прѣзъ 1899.
Той се явява поправка на Дриновския правописъ и остава въ сила до въвеждането на Омарчевския правописъ прѣзъ 1921. 
Иванчевскиятъ правописъ се характеризира съ:
%%%%%%%%%%
%
%%%%%%%%%%
% Thesaurus checking
% народъ
\begin{itemize}
	\item прѣдставяне на звука [ъ] съ буквитѣ \textbf{ъ} и \textbf{ѫ} (голѣма носовка), като различието се прави на етимологическа основа.
	\item употрѣба на буквата \textbf{ѣ} (ятъ) на етимологическото ѝ мѣсто.
	\item употрѣба на краесловни ерове.
	\item синтактично правило за писане на пъленъ и кратъкъ членъ.
	\item писане на групитѣ ър/ръ и ъл/лъ по североизточното произношение.
	\item писане на глаголнитѣ окончания съ а и я.
\end{itemize}
%%%%%%%%%%


Ние смѣтаме, че правописната реформа отъ 1945-та съ прѣмахването на буквитѣ ятъ („ѣ“) и голѣмъ юсъ („ѫ“) отъ азбуката е въ грубъ разрѣзъ съ писмената ни традиция, нарушава единството на говоритѣ въ книжовния езикъ и затруднява приемственостьта между поколѣнията читатели.

Иванчевскиятъ правописъ прѣмахва дотогавашната употрѣба (т.е. съгласно Дриновския правописъ) на Ѫ и Ѭ въ слѣднитѣ случаи:

\begin{itemize}
	\item Въ пълната и кратка винителна форма на личното мѣстоимение въ 3 л. ед. ч. ж. р.: неѭ, ѭ
	
	\item Въ 1 л. ед. ч. сег. вр. на глаголитѣ отъ I и II спрежение, напримѣръ: азъ четѫ, азъ пишѫ, азъ играѭ, азъ мълчѫ, азъ молѭ, азъ вървѭ
	
	\item Въ 3 л. мн. ч. сег. вр. на глаголитѣ отъ всички спрежения, напримѣръ: тѣ четѫтъ, тѣ пишѫтъ, тѣ играѭтъ, тѣ мълчѫтъ, тѣ молѭтъ, тѣ вървѭтъ, тѣ искѫтъ, тѣ излитѫтъ, тѣ отивѫтъ
	
	\item Въ 3 л. мн. ч. мин. св. вр. и мин. несв. вр. на глаголитѣ, напримѣръ: тѣ четохѫ, тѣ пишехѫ, тѣ играехѫ, тѣ мълчахѫ, тѣ молихѫ, тѣ вървѣхѫ, тѣ искахѫ, тѣ излитахѫ, тѣ отивахѫ
\end{itemize}

 

\end{document}
