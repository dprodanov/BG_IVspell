% !TeX encoding = UTF-8
% !TeX spellcheck = bg_BG_IV
 
\documentclass{article}
%\usepackage[T2A, T1]{fontenc}
%\usepackage[utf8]{inputenc}
 
 \usepackage{fontspec}
 \setmainfont{FreeSerif}
 \setsansfont{FreeSans}
 \setmonofont{FreeMono}
% \setmainfont{"DejaVu"}
 %\setsansfont{"DejaVu Sans"}
 %\setmonofont{"DejaVu Sans Mono"}
 
\usepackage{polyglossia}
\setdefaultlanguage{bulgarian}

%
% http://ivanchevski.grazhdani.eu/slovnik.html
% Thesaurus check - the brown fox is running. 
%%%%%%%%%%%
%  Doc
%%%%%%%%%%%
\begin{document}
	%%%%%%%%%%%%%
	%  Title
	%%%%%%%%%%%%%
	\title{Иванчевски правописъ}
	\maketitle
	
	Речникъ на българския езикъ, съдържащъ слова и тѣхнитѣ форми споредъ изискванията на Иванчевския правописъ. 
	Иванчевски правописъ е името, давано понѣкога на първиятъ официаленъ български правописъ, 
	въвѣденъ съ наредба на министъра на народното просвѣщение Тодоръ Иванчовъ прѣзъ 1899.
 	Той се явява поправка на Дриновския правописъ и остава въ сила до въвеждането на Омарчевския правописъ прѣзъ 1921. 
 	Иванчевскиятъ правописъ се характеризира съ:
 	%%%%%%%%%%
 	%
 	%%%%%%%%%%
 	% Thesaurus checking
 	% народъ
 	\begin{itemize}
 		\item прѣдставяне на звука [ъ] съ буквитѣ \textbf{ъ} и \textbf{ѫ} (голѣма носовка), като различието се прави на етимологическа основа.
 		\item употрѣба на буквата \textbf{ѣ} (ятъ) на етимологическото ѝ мѣсто.
 		\item употрѣба на краесловни ерове.
 		\item синтактично правило за писане на пъленъ и кратъкъ членъ.
 		\item писане на групитѣ ър/ръ и ъл/лъ по североизточното произношение.
 		\item писане на глаголнитѣ окончания съ а и я.
 	\end{itemize}
    %%%%%%%%%%
    
    
    Ние смѣтаме, че правописната реформа отъ 1945-та съ прѣмахването на буквитѣ ять („ѣ“) и голѣмъ юсъ („ѫ“) отъ азбуката е въ грубъ разрѣзъ съ писмената ни традиция, нарушава единството на говоритѣ въ книжовния езикъ и затруднява приемственостьта между поколѣнията читатели.
 
%%%%%%%%%%
\end{document}