% !TeX encoding = UTF-8
% !TeX spellcheck = bg_BG_IV
\documentclass{article}
\usepackage[X2]{fontenc}
\usepackage{newunicodechar}
%\DeclareUnicodeCharacter{0466}{\CYRLYUS}
%\DeclareUnicodeCharacter{0467}{\cyrlyus}

% iotated characters
%\DeclareUnicodeCharacter{046C}{\CYRII\CYRBYUS}
%\DeclareUnicodeCharacter{046D}{\cyrii\cyrbyus}



\usepackage[bulgarian]{babel}
%\usepackage{cmsrb}

%\newcommand{\YUSligature}{s}
%\newunicodechar{ѭ}{\YUSligature}
\newunicodechar{й}{i}
\newunicodechar{і}{\cyrii}
%opening
\title{Стъклени стени}
\author{Лили Иванова}

\begin{document}

\maketitle

 \textit{Твойтѣ} рѫцѣ паліѫтъ нощьта,\\
 трепетъ и страсть ще се случатъ, може би. \\
 Моето „не“, твоето „да“ -- \\
 всичко това вѣроятно прѣдстои. \\
 
 Разпилѣй докрай душата ми, \\
 но дали ще си сѫдбата ми? \\
 
 Може би днесъ, може би тукъ, \\
 може би утрѣ призори, \\
 може би съ тебъ, може би съ менъ \\
 ще стане чудо. \\
 
 Може би азъ, може би ти, \\
 скрити задъ стъклени стени, \\
 съ прѣплетени души, \\
 пакъ ще сме чужди. \\
 
 Може би днесъ, може би тукъ, \\
 може би утрѣ призори, \\
 може би съ тебъ, може би съ менъ \\
 ще стане чудо. \\
 
 Може би въ насъ тя ще стопи \\
 тежкитѣ стъклени стени. \\
 Ела, любовь, \\
 не искамъ друго! \\
 
 Твойтѣ рѫцѣ палiѫт нощьта, \\
 трепетъ и страсть ще се случатъ, може би. \\
 Моето „не“, твоето „да“ -- \\
 всичко това вѣроятно прѣдстои. \\
 
 Разпилѣй докрай душата ми, \\
 но дали ще си сѫдбата ми? \\
 
 Може би днесъ, може би тукъ, \\
 може би утрѣ призори, \\
 може би съ тебъ, може би с менъ \\
 ще стане чудо. \\
 
 Може би аз, може би ти, \\
 скрити задъ стъклени стени, \\
 с прѣплетени души, \\
 пакъ ще сме чужди. \\
 
 Може би днесъ, може би тукъ, \\
 може би утрѣ призори, \\
 може би съ тебъ, може би съ менъ \\
 ще стане чудо. \\
 
 Може би въ насъ тя ще стопи \\
 тежкитѣ стъклени стени. \\
 Ела, любовь, \\
 не искамъ друго! \\
 
 Ела, любовь,  \\
 не искамъ друго! \\

\end{document}
