% !TeX encoding = UTF-8
% !TeX spellcheck = bg_BG_IV
 
\documentclass{article}
%\usepackage[T2A, T1]{fontenc}
%\usepackage[utf8]{inputenc}
 
 \usepackage{fontspec}
 \setmainfont{FreeSerif}
 \setsansfont{FreeSans}
 \setmonofont{FreeMono}

\usepackage{polyglossia}
\setdefaultlanguage{bulgarian}

%
% http://ivanchevski.grazhdani.eu/slovnik.html
% 
%%%%%%%%%%%
%  Doc
%%%%%%%%%%%
\begin{document}
	%%%%%%%%%%%%%
	%  Title
	%%%%%%%%%%%%%
	\title{Употрѣба на Ѫ}
	\maketitle
	    
	Буквата Ѫ, наричана голѣмъ юсъ или голѣма носовка се пише на етимологиченъ принципъ въ слѣднитѣ думи:
 
 	%%%%%%%%%%
 	%
 	%%%%%%%%%%
 	\begin{description}
 		\item [Коренъ -БѪБР-]: бѫбрековиденъ, бѫбрекъ, бѫбрече, бѫбреченъ, надбѫбреченъ, околобѫбреченъ
 		\item[Коренъ -БѪД-]: безбѫдещность, бѫда, бѫдатъ, бѫде, бѫдейки, бѫделъ, бѫдемъ, бѫденъ, бѫдете, бѫдеха, бѫдехме, бѫдехте, бѫдехъ, бѫдеше, бѫдешъ, бѫдеще, бѫдещность, бѫдещъ, бѫди, бѫдникъ, бѫднина, несбѫдналъ, несбѫднатъ, прѣбѫда, прѣбѫдвамъ, прѣбѫдване, сбѫдвамъ, сбѫдване, сбѫдна, сбѫднатъ
 		
 		\item[Коренъ -ВѪБ-]: вѫбелъ
 		
 		\item[Коренъ -ВѪГ-]: вѫгарецъ
 		
 		\item[Коренъ -ВѪГЛ-]: вѫглеводородъ, вѫгледвуокисъ, вѫгледобивъ, вѫглекопачъ, вѫгленочеренъ, вѫгленъ, вѫглероденъ, вѫглеродъ, вѫглехидратъ, вѫглища, вѫглищарь, вѫглищенъ, каменовѫгленъ, обезвѫгляване, овѫгленъ, овѫгля, овѫглявамъ, овѫгляване, сѣровѫглеродъ
 		\item[Коренъ -ВѪД- / -ВѪЖД-] : вѫдица, вѫдичарски, вѫдичарство, вѫдичарь, вѫдичка, вѫдя, говедовѫденъ, говедовѫдство, говедовѫдъ, животновѫденъ, животновѫдка, животновѫдство, животновѫдъ, завѫдя, завѫждамъ, завѫждане, извѫдя, извѫждамъ, извѫждане, коневѫденъ, коневѫдецъ, коневѫдство, коневѫдъ, лесовѫденъ, лесовѫдство, лесовѫдъ, навѫдя, навѫждамъ, навѫждане, овцевѫденъ, овцевѫдецъ, овцевѫдство, овцевѫдъ, отвѫденъ, отвѫдморски, отвѫдокеански, отвѫдъ, отвѫждамъ, полевѫденъ, полевѫдство, полевѫдъ, птицевѫденъ, птицевѫдка, птицевѫдство, птицевѫдъ, пчеловѫдка, пчеловѫдски, пчеловѫдство, пчеловѫдъ, развѫденъ, развѫдникъ, развѫдя, развѫждамъ, развѫждане, растениевѫдство, растениевѫдъ, рибовѫденъ, рибовѫдство, рибовѫдъ, свиневѫдецъ, свиневѫдка, свиневѫдство, свиневѫдъ, скотовѫденъ, скотовѫдецъ, скотовѫдство, скотовѫдъ

    	\item[Коренъ -ВѪЗ- / -ВѪЖ-]: вѫжарски, вѫжарь, вѫже, вѫжеиграчество, вѫжеиграчка, вѫжеиграчъ, вѫженце, вѫженъ, вѫзелъ, вѫзлестъ, вѫзловатъ, вѫзловъ, хидровѫзелъ

		\item[Коренъ -ВѪС-]: вѫсене, вѫсеница, вѫси, вѫся, навѫсвамъ, навѫсване, навѫсено, навѫсеность, навѫсенъ, навѫся, свѫсвамъ, свѫсване, свѫсено, свѫсенъ, свѫся
		
	\item[Коренъ -ВѪТ-]: вѫтъкъ, вѫтъченъ
	
	\item[Коренъ -ВѪТР-]: вѫтрѣшенъ, вѫтрѣшно, вѫтрѣшнопартиенъ, вѫтрѣшнополитически, вѫтрѣшность, вѫтрѣшноядренъ, вѫтрѣ, извѫтрѣ, навѫтрѣ, отвѫтрѣ
	
	\item[Коренъ -ГЛѪБ-]: вглѫбеность, вглѫбя, вглѫбявамъ, вглѫбяване, глѫбина, углѫбя, углѫбявамъ, углѫбяване
	
	\item[Коренъ -ГНѪС-]: гнѫсене, гнѫсенъ, гнѫсливость, гнѫсливъ, гнѫсно, гнѫсота, гнѫсотия, гнѫсь, гнѫся, догнѫсявамъ, догнѫсяване, погнѫса, погнѫсено, погнѫся, погнѫсявамъ, погнѫсяване
	
	\item[Коренъ -ГЪЛѪБ-]: гълѫбарникъ, гълѫбарь, гълѫбица, гълѫбовъ, гълѫбче, гълѫбъ
	
	\item[Коренъ -ГѪБ-]: гѫба, гѫбавъ, гѫбарникъ, гѫбарство, гѫбарь, гѫбенъ, гѫбестъ, гѫбичка, гѫбовиденъ, гѫбообразенъ, гѫбясамъ, гѫбясвамъ, гѫбясване
	
	\item[Коренъ -ГѪГ-]: гѫгна, гѫгнене, гѫгнивъ, гѫгрица
	
	\item[Коренъ -ГѪД-]: гѫдуларь, гѫдулка
	
	\item[Коренъ -ГѪЖ-]: гѫжва
	
	\item[Коренъ -ГѪЗ-]: гѫзарь, гѫзаря, гѫзарѣя, гѫзе, гѫзенце, гѫзище, гѫзоблизецъ, гѫзурча, гѫзче, гѫзъ, изгѫзаря, изгѫзарявамъ, изгѫзарѣя, изгѫзвамъ, изгѫзя, нагѫзарѣя, нагѫзвамъ, нагѫзурчвамъ, нагѫзурчване, нагѫзя, огѫзвамъ, огѫзя, разгологѫзвамъ, разгологѫзване, разгѫзвамъ, разгѫзя
	
	\item[Коренъ -ГѪС- / -ГѪШ-]: гѫсарка, гѫсарь, гѫсе, гѫсеница, гѫсеничавъ, гѫсениченъ, гѫска, гѫсокъ, гѫши
	
	\item[Коренъ -ГѪСТ-]: гѫстакъ, гѫсталакъ, гѫстичъкъ, гѫстонаселенъ, гѫстообрасълъ, гѫстота, гѫстъ, гѫстѣя, несгѫстимость, несгѫстимъ, посгѫстя, посгѫстявамъ, посгѫстяване, сгѫстено, сгѫстенъ, сгѫстимость, сгѫстителенъ, сгѫститель, сгѫстя, сгѫстявамъ, сгѫстяване
	
	\item[Коренъ -ДѪБ-]: дѫбакъ, дѫбиленъ, дѫбовъ, дѫбрава, дѫбъ
	
	\item[Коренъ -ДѪГ-]: дѫга, дѫговиденъ, дѫговидно, дѫговъ, дѫгообразенъ, недѫгавость, недѫгавъ, недѫгъ
	
	\item[Коренъ -ЖЬЛѪД-]: желѫдовиденъ, желѫдъ, жълѫдовъ, жълѫдъ
	
	\item[Коренъ -ЗѪБ-]: беззѫбъ, зѫбарь, зѫбатъ, зѫбене, зѫбенъ, зѫберъ, зѫбестъ, зѫбецъ, зѫбоболъ, зѫбовиденъ, зѫболѣкарка, зѫболѣкарски, зѫболѣкарство, зѫболѣкарь, зѫболѣчебенъ, зѫболѣчебница, зѫболѣчение, зѫботехника, зѫботехникъ, зѫбчатка, зѫбчатъ, зѫбче, зѫбъ, зѫбя, назѫбвамъ, назѫбване, назѫбеность, назѫбенъ, назѫбя, назѫбямъ, назѫбяне, неназѫбенъ, озѫбвамъ, озѫбване, озѫбя, острозѫбъ, тризѫбецъ
	
	\item[Коренъ -КРѪГ- / -КРѪЖ-]: закрѫгленичъкъ, закрѫглено, закрѫгленость, закрѫгленъ, закрѫгля, закрѫглявамъ, закрѫгляване, закрѫглямъ, закрѫгляне, крѫгличъкъ, крѫгло, крѫглоликъ, крѫговиденъ, крѫговратенъ, крѫговратъ, крѫговъ, крѫгозоръ, крѫгомъ, крѫгообразенъ, крѫгообращение, крѫгче, крѫгъ, крѫгълъ, крѫжа, крѫжене, крѫжецъ, крѫжило, крѫжокъ, крѫжоченъ, крѫжочникъ, крѫжочничка, обкрѫжа, обкрѫжавамъ, обкрѫжаване, обкрѫжение, окрѫгля, окрѫглявамъ, окрѫгляване, окрѫгъ, окрѫжавамъ, окрѫжаване, окрѫженъ, окрѫжие, окрѫжность, позакрѫгля, позакрѫглявамъ, позакрѫгляване, позакрѫглямъ, позакрѫгляне, политкрѫжокъ, полукрѫгъ, полукрѫгълъ, полуокрѫжность
	
	\item[Коренъ -КѪД-]: всѣкѫдѣ, докѫдѣ, докѫдѣто, доникѫдѣ, донѣкѫдѣ, кѫдѣ, кѫдѣто, навсѣкѫдѣ, накѫдѣ, накѫдѣто, нанѣкѫдѣ, никѫдѣ, нѣкѫдѣ, нѣкѫдѣшенъ, отвсѣкѫдѣ, откѫдѣ, откѫдѣто, отникѫдѣ, отнѣкѫдѣ
	
	\item[Коренъ -КѪДР-]: закѫдря, закѫдрямъ, закѫдряне, кѫдравость, кѫдравъ, кѫдрене, кѫдрица, кѫдрокосъ, кѫдря, накѫдря, накѫдрямъ, накѫдряне, разкѫдрямъ, разкѫдряне
	\item[Коренъ -КѪК-]: кѫклица, кѫкличавъ
	
	\item[Коренъ -КѪП-]: изкѫпвамъ, изкѫпване, изкѫпя, кѫпаленъ, кѫпалня, кѫпане, кѫпина, кѫпинакъ, кѫпиновъ, кѫпона, кѫпя, неокѫпанъ, окѫпвамъ, окѫпване, окѫпя, поизкѫпвамъ, поизкѫпване, поизкѫпя
	
	\item[Коренъ -КѪС-]: възкѫсъ, докѫсвамъ, докѫсване, закѫсамъ, закѫсвамъ, закѫсване, изкѫсамъ, изкѫсане, изкѫсвамъ, изкѫсване, изкѫсо, изпокѫсамъ, изпокѫсане, изпокѫсанъ, изпокѫсвамъ, изпокѫсване, изпонакѫсвамъ, изпонакѫсване, кѫсамъ, кѫсане, кѫсване, кѫсенъ, кѫсичко, кѫсичъкъ, кѫсно, кѫсо, кѫсовратъ, кѫсовълновъ, кѫсогледство, кѫсогледъ, кѫсокракъ, кѫсометраженъ, кѫсопаметенъ, кѫсопаметство, кѫсоподстриганъ, кѫсота, кѫсоуменъ, кѫсоцевенъ, кѫсче, кѫсъ, кѫсъ, накѫсамъ, накѫсане, накѫсвамъ, накѫсване, накѫсо, ненакѫсанъ, неоткѫснатъ, непрѣкѫсваемъ, непрѣкѫсванъ, непрѣкѫсващъ, непрѣкѫснато, непрѣкѫснатость, непрѣкѫснатъ, неразкѫсанъ, неразкѫсваемъ, нескѫсенъ, окѫсамъ, окѫсане, окѫсанъ, окѫсвамъ, окѫсване, окѫсявамъ, откѫсамъ, откѫсане, откѫсвамъ, откѫсване, откѫслекъ, откѫслеченъ, откѫслечно, откѫслечность, откѫсна, откѫснато, откѫснатость, откѫсъ, позакѫсамъ, позакѫсане, позакѫсвамъ, позакѫсване, покѫсамъ, покѫсане, полуразкѫсамъ, поокѫсамъ, поокѫсане, поокѫся, поокѫсявамъ, поокѫсяване, поокѫсѣя, поскѫсамъ, поскѫсане, поскѫся, поскѫсявамъ, поскѫсяване, прокѫсамъ, прокѫсане, прокѫсвамъ, прокѫсване, прѣкѫсваемъ, прѣкѫсвамъ, прѣкѫсване, прѣкѫсванъ, прѣкѫсвачъ, прѣкѫсна, разкѫсамъ, разкѫсане, разкѫсаность, разкѫсвамъ, разкѫсване, разпокѫсамъ, разпокѫсане, разпокѫсаность, разпокѫсвамъ, разпокѫсване, скѫсамъ, скѫсане, скѫсанъ, скѫсвамъ, скѫсване, скѫсенъ, скѫся, скѫсявамъ, скѫсяване, ултракѫсъ
	
	\item[Коренъ -КѪТ-]: закѫтамъ, закѫтано, закѫтанъ, закѫтвамъ, закѫтване, кѫтамъ, кѫтане, кѫтенъ, кѫткамъ, кѫткане, кѫтникъ, кѫтче, кѫтъ, непокѫтнатость, непокѫтнатъ, позакѫтвамъ, позакѫтване, покѫтвамъ, покѫтване, покѫтна, поприкѫтвамъ, поприкѫтване, поскѫтвамъ, прикѫтамъ, прикѫтане, прикѫтвамъ, прикѫтване, скѫтамъ, скѫтане, скѫтвамъ, скѫтване
	
	\item[Коренъ -КѪЩ-]: вкѫщи, въртокѫщникъ, въртокѫщница, кѫща, кѫщенъ, кѫщица, кѫщичка, кѫщникъ, кѫщовенъ, кѫщовникъ, кѫщовница, кѫщовничество, кѫщурка, покѫщнина
	
	\item[Коренъ -ЛѪГ-]: лѫгъ
	
	\item[Коренъ -ЛѪК- / -ЛѪЧ-]: безотлѫченъ, излѫча, излѫчвамъ, излѫчване, лѫка, лѫкавецъ, лѫкаво, лѫкавость, лѫкавствамъ, лѫкавстване, лѫкавство, лѫкавщина, лѫкавъ, лѫкатуша, лѫкатушене, лѫкатушенъ, лѫкатушно, лѫковиденъ, лѫковъ, лѫкъ, лѫча, лѫчевъ, лѫчезаренъ, лѫчезарно, лѫчезарность, Лѫчезаровъ, Лѫчезаръ, лѫчеизпускане, лѫчение, лѫчеобразенъ, лѫчистъ, Лѫчо, лѫчъ, неизлѫчващъ, неотлѫченъ, неотлѫчно, неразлѫченъ, неразлѫчно, неразлѫчность, облѫча, облѫчвамъ, облѫчване, отлѫча, отлѫчвамъ, отлѫчване, отлѫчка, петолѫченъ, петолѫчка, поотлѫчвамъ, поотлѫчване, разлѫка, разлѫча, разлѫчвамъ, разлѫчване, самоотлѫча, самоотлѫчвамъ, самоотлѫчване, самоотлѫчка, шестолѫченъ
	
	\item[Коренъ -МОГѪЩ-]: всемогѫщество, всемогѫщъ, могѫщественъ, могѫщество, могѫщъ, свръхмогѫщъ
	
	\item[Коренъ -МѪДъР-]: всемѫдъръ, измѫдрувамъ, измѫдруване, измѫдря, измѫдрямъ, измѫдряне, мѫдрене, мѫдрецъ, мѫдро, мѫдрость, мѫдрувамъ, мѫдруване, мѫдря, мѫдрѣя, мѫдъръ, намѫдрувамъ, намѫдруване, намѫдря, намѫдрямъ, нецѣломѫдреность, нецѣломѫдренъ, нецѣломѫдрие, помѫдрявамъ, помѫдряване, прѣмѫдро, прѣмѫдрость, прѣмѫдъръ, умѫдря, умѫдрявамъ, умѫдряване, умѫдряне, умѫдрѣя, цѣломѫдрено, цѣломѫдреность, цѣломѫдренъ, цѣломѫдрие
	
	\item[Коренъ -МѪЖ-]: възмѫжавамъ, възмѫжаване, възмѫжалость, възмѫжея, мѫжага, мѫжемразка, мѫжественость, мѫжественъ, мѫжество, мѫжеубийство, мѫжеубийца, мѫжкарана, мѫжкарка, мѫжкарски, мѫжкарь, мѫжки, мѫжле, мѫжовъ, мѫжъ, немѫжественъ, неомѫженъ, новоомѫженъ, омѫжа, омѫжвамъ, омѫжване
	
	\item[Коренъ -МѪК- / -МѪЧ-]: великомѫченикъ, великомѫченица, домѫчнявамъ, домѫчнѣя, замѫча, измѫча, измѫчвамъ, измѫчване, измѫченъ, мѫка, мѫча, мѫчене, мѫчение, мѫченикъ, мѫченица, мѫченически, мѫченичество, мѫченъ, мѫчилище, мѫчителенъ, мѫчителка, мѫчително, мѫчителски, мѫчитель, мѫчно, мѫчнодостѫпенъ, мѫчноопрѣдѣлимъ, мѫчноподвиженъ, мѫчноподвижность, мѫчноприспособимъ, мѫчнопроходимъ, мѫчнопрѣодолимъ, мѫчноразтопимъ, мѫчносмилаемъ, мѫчносмилателенъ, мѫчнотия, мѫчнотопимость, мѫчнѣя, намѫча, намѫчвамъ, намѫчване, поизмѫча, поизмѫчвамъ, поизмѫчване, помѫча, помѫчвамъ, помѫчване, първомѫченикъ, самоизмѫча, самоизмѫчвамъ, самоизмѫчване, умѫчненъ
	
	\item[Коренъ -МѪТ-]: замѫтвамъ, замѫтване, замѫтенъ, замѫтя, измѫтвамъ, измѫтване, измѫтя, мѫтене, мѫтеница, мѫтенъ, мѫтилка, мѫтило, мѫтно, мѫтность, мѫтнѣя, мѫтя, помѫтвамъ, помѫтване, помѫтненъ, помѫтнявамъ, помѫтняване, помѫтнѣя, помѫтя, размѫтвамъ, размѫтване, размѫтя, смѫтенъ, смѫтно, смѫтность, смѫтолевя, смѫтолевямъ, смѫтолевяне
	
	\item[Коренъ -ПРѪТ- / -ПРѪЧ-]: прѫтъ, прѫчица, прѫчка, прѫчковиденъ
	
	\item[Коренъ -ПѪД- / -ПѪЖД-]: изпѫденъ, изпѫдя, изпѫждамъ, изпѫждане, напѫдя, напѫждамъ, напѫждане, отпѫдя, отпѫждамъ, отпѫждане, пропѫдя, пропѫждамъ, пропѫждане, пѫдаринъ, пѫдарски, пѫдарь, пѫдя, разпѫдя, разпѫждамъ, разпѫждане
	
	\item[Коренъ -ПѪК- / -ПѪЧ-]: изпѫквамъ, изпѫкване, изпѫкна, изпѫкналость, изпѫкналъ, изпѫкнатина, изпѫча, изпѫчвамъ, изпѫчване, пѫкна, пѫча, пѫчене
	
	\item[Коренъ -ПѪП-]: напѫпвамъ, напѫпване, напѫпя, понапѫпвамъ, понапѫпване, понапѫпя, пѫпенъ, пѫпешовъ, пѫпешъ, пѫпка, пѫпковиденъ, пѫпкувамъ, пѫпкуване, пѫпчасамъ, пѫпчасвамъ, пѫпчасване, пѫпчестъ, пѫпчивость, пѫпчивъ, пѫпчица, пѫпъ, разпѫпвамъ, разпѫпване, разпѫпя, разпѫпямъ, разпѫпяне, сѣмепѫпка
	
	\item[Коренъ -ПѪТ-]: безпѫтенъ, безпѫтие, безпѫтица, безпѫтно, безпѫтность, безпѫтство, желѣзопѫтенъ, запѫтвамъ, запѫтване, запѫтя, крайпѫтенъ, кръстопѫть, напѫтвамъ, напѫтване, напѫтствамъ, напѫтстване, напѫтственикъ, напѫтственъ, напѫтствие, напѫтствувамъ, напѫтствуване, напѫтя, отпѫтувамъ, отпѫтуване, подпѫтвамъ, подпѫтване, попѫтенъ, попѫтувамъ, попѫтуване, пропѫтувамъ, пропѫтуване, противоспѫтниковъ, пѫтеводенъ, пѫтеводитель, пѫтека, пѫтемъ, пѫтенъ, пѫтеписецъ, пѫтеписъ, пѫтепоказатель, пѫтеуказатель, пѫтечка, пѫтешествамъ, пѫтешестване, пѫтешественикъ, пѫтешественица, пѫтешествие, пѫтешествувамъ, пѫтешествуване, пѫти, пѫтникъ, пѫтница, пѫтнически, пѫтничка, пѫтнишки, пѫтнопаваженъ, пѫтностроителенъ, пѫтнотранспортенъ, пѫтувамъ, пѫтуване, пѫть, пѫтьомъ, разпѫтенъ, разпѫтъ, спѫтниковъ, спѫтникъ, спѫтница, спѫтничка, сухопѫтенъ, сухопѫтно, съпѫтникъ, съпѫтница, съпѫтствамъ, съпѫтстване, съпѫтствувамъ, съпѫтствуване, упѫтвамъ, упѫтване, упѫтя
	
	\item[Коренъ -РѪБ-]: изрѫбвамъ, изрѫбване, изрѫбенъ, изрѫбя, нарѫбване, нарѫбенъ, нарѫбя, непорѫбенъ, подрѫбвамъ, подрѫбване, подрѫбя, подрѫбямъ, подрѫбяне, порѫбвамъ, порѫбване, порѫбя, порѫбямъ, порѫбяне, рѫбатость, рѫбатъ, рѫбестъ, рѫбъ, рѫбя
	
	\item[Коренъ -РѪГ- / -РѪД- / -РѪЖ-]:въорѫжа, въорѫжавамъ, въорѫжаване, въорѫжение, въорѫженость, въорѫженъ, лековъорѫженъ, невъорѫженъ, обезорѫжа, обезорѫжавамъ, обезорѫжаване, орѫдеенъ, орѫжеенъ, орѫжейникъ, орѫженосецъ, орѫдие, орѫжие, прѣвъорѫжа, прѣвъорѫжавамъ, прѣвъорѫжаване, разорѫжа, разорѫжавамъ, разорѫжаване, разорѫжение, съорѫжа, съорѫжавамъ, съорѫжаване, съорѫжение, съпружески, съпружество, тежковъорѫженъ, хорѫгва
	
	\item[Коренъ -РѪК- / -РѪЧ-]: безрѫкъ, бързорѫкъ, взаимопорѫчитель, врѫча, врѫчвамъ, врѫчване, двурѫкъ, дългорѫкъ, еднорѫкъ, зарѫка, зарѫкомахамъ, зарѫкоплѣскамъ, зарѫчамъ, зарѫчвамъ, зарѫчване, зарѫчение, зарѫчка, изрѫкоплѣскамъ, изрѫкоплѣскане, изрѫчкамъ, изрѫчквамъ, изрѫчкване, нарѫкавникъ, нарѫки, нарѫченъ, нарѫчникъ, нарѫчъ, непорѫчанъ, непрѣпорѫчителенъ, непрѣпорѫчителность, нерѫкотворенъ, несрѫченъ, несрѫчно, несрѫчность, обрѫчъ, отрѫки, подрѫка, подрѫченъ, порѫчамъ, порѫчане, порѫчвамъ, порѫчване, порѫчение, порѫчителка, порѫчителствамъ, порѫчителстване, порѫчителство, порѫчителствувамъ, порѫчителствуване, порѫчитель, порѫчка, прѣпорѫка, прѣпорѫчамъ, прѣпорѫчане, прѣпорѫчано, прѣпорѫчанъ, прѣпорѫчвамъ, прѣпорѫчване, прѣпорѫчителенъ, прѣпорѫчителка, прѣпорѫчителность, прѣпорѫчитель, прѫчица, прѫчка, прѫчковиденъ, рѫка, рѫкавелъ, рѫкавенъ, рѫкавица, рѫкавичарь, рѫкавъ, рѫкатка, рѫководене, рѫководенъ, рѫководещъ, рѫководителка, рѫководителски, рѫководитель, рѫководство, рѫководя, рѫкодѣлие, рѫкойка, рѫкомахамъ, рѫкомахане, рѫкопашенъ, рѫкописенъ, рѫкописно, рѫкописъ, рѫкоплѣскамъ, рѫкоплѣскане, рѫкополагамъ, рѫкополагане, рѫкоположа, рѫкостискамъ, рѫкостискане, рѫкохватка, рѫкувамъ, рѫкуване, рѫченикъ, рѫченица, рѫченъ, рѫчичка, рѫчка, рѫчкамъ, рѫчкане, рѫчна, рѫчно, саморѫченъ, саморѫчно, собственорѫченъ, собственорѫчно, срѫченъ, срѫчкамъ, срѫчкане, срѫчквамъ, срѫчкване, срѫчно, срѫчность
	
	\item[Коренъ -СКѪД-]: оскѫденъ, оскѫдица, оскѫдность
	
	\item[Коренъ -СКѪП-]: нескѫпоцененъ, оскѫпя, оскѫпявамъ, оскѫпяване, поскѫпвамъ, поскѫпване, поскѫпна, поскѫпнѣя, поскѫпя, поскѫпявамъ, поскѫпяване, прѣскѫпо, прѣскѫпъ,
	 скѫпене, скѫперникъ, скѫперница, скѫпернически, скѫперничество, скѫпо, скѫпотия, скѫпоцененъ, скѫпоценность, скѫпчийка, скѫпчия, скѫпъ, скѫпя, скѫпѣя
	
	\item[Коренъ -СТѪП-]: богоотстѫпникъ, богоотстѫпнически, богоотстѫпничество, военнопрѣстѫпникъ, встѫпвамъ, встѫпване, встѫпителенъ, встѫпление, встѫпя, встѫпямъ, вѣроотстѫпникъ, вѣроотстѫпница, вѣроотстѫпнически, вѣроотстѫпничество, двустѫпенъ, достѫпенъ, достѫпно, достѫпность, достѫпъ, заотстѫпвамъ, заотстѫпване, запристѫпвамъ, застѫпвамъ, застѫпване, застѫпникъ, застѫпница, застѫпнически, застѫпничество, застѫпя, застѫпямъ, застѫпяне, изстѫпвамъ, изстѫпване, изстѫпление, изстѫпленъ, изстѫпя, изстѫпямъ, изстѫпяне, клетвопрѣстѫпникъ, клетвопрѣстѫпница, клетвопрѣстѫпнически, клетвопрѣстѫпничество, контранастѫпление, леснодостѫпенъ, леснодостѫпно, многостѫпаленъ, мѣстопрѣстѫпление, настѫпателенъ, настѫпателно, настѫпателность, настѫпвамъ, настѫпване, настѫпление, настѫпя, настѫпямъ, настѫпяне, невстѫпилъ, недостѫпенъ, недостѫпность, неотстѫпващъ, неотстѫпенъ, неотстѫпно, неотстѫпность, неотстѫпчиво, неотстѫпчивость, неотстѫпчивъ, непристѫпенъ, непристѫпность, новопостѫпилъ, общодостѫпенъ, общодостѫпность, отстѫпателенъ, отстѫпателно, отстѫпвамъ, отстѫпване, отстѫпителенъ, отстѫпка, отстѫпление, отстѫпникъ, отстѫпница, отстѫпничество, отстѫпчиво, отстѫпчивость, отстѫпчивъ, отстѫпъ, отстѫпя, подстѫпъ, поотстѫпя, постѫпателенъ, постѫпателно, постѫпвамъ, постѫпване, постѫпка, постѫпление, постѫпя, пристѫпвамъ, пристѫпване, пристѫпъ, пристѫпя, пристѫпямъ, пристѫпяне, простѫпвамъ, простѫпване, простѫпка, простѫпя, прѣотстѫпвамъ, прѣотстѫпване, прѣотстѫпя, прѣотстѫпяне, прѣстѫпвамъ, прѣстѫпване, прѣстѫпенъ, прѣстѫпенъ, прѣстѫпление, прѣстѫпникъ, прѣстѫпница, прѣстѫпнически, прѣстѫпно, прѣстѫпность, прѣстѫпя, прѣстѫпямъ, прѣстѫпяне, родоотстѫпникъ, родоотстѫпница, родоотстѫпнически, родоотстѫпничество, стѫпало, стѫпаловиденъ, стѫпвамъ, стѫпване, стѫписамъ, стѫписане, стѫписвамъ, стѫписване, стѫпка, стѫпканъ, стѫпквамъ, стѫпкване, стѫпна, стѫпча, стѫпя, тристѫпенъ, труднодостѫпенъ
	Думата СѪ: 3 л. мн. ч. на глагола съмъ, напр. Тѣ сѫ пристигнали. Паритѣ сѫ откраднати.
	
	\item[Коренъ -СѪБ-]: Сѫбевъ, Сѫби, Сѫбковъ, сѫбота, сѫботенъ, Сѫботиновъ, Сѫботинъ, Сѫбчевъ, сѫботянинъ, сѫботянка, сѫботянски, сѫботянство
	
	\item[Коренъ -СѪД- / -СѪЖД-]: безразсѫденъ, безразсѫдно, безразсѫдность, безразсѫдство, безразсѫдствувамъ, безразсѫдствуване, безсѫдържателенъ, безсѫдържателность, военносѫдебенъ, извънсѫдебенъ, насѫдя, необсѫденъ, неосѫденъ, неосѫжданъ, неподсѫденъ, неподсѫдность, неразсѫденъ, неразсѫдителенъ, неразсѫдливо, неразсѫдливость, неразсѫдливъ, несѫденъ, обсѫдя, обсѫждамъ, обсѫждане, осѫденъ, осѫдителенъ, осѫдително, осѫдителность, осѫдя, осѫждамъ, осѫждане, отсѫдя, отсѫждамъ, отсѫждане, подсѫденъ, подсѫдимость, подсѫдимъ, подсѫдно, подсѫдность, пообсѫждамъ, пообсѫждане, правосѫденъ, правосѫдие, присѫда, присѫдя, присѫждамъ, присѫждане, прѣдразсѫдъкъ, прѣдразсѫдъченъ, разсѫдителенъ, разсѫдително, разсѫдителность, разсѫдливо, разсѫдливость, разсѫдливъ, разсѫдъкъ, разсѫдъченъ, разсѫдъчно, разсѫдъчность, разсѫдя, разсѫждавамъ, разсѫждаване, разсѫждамъ, разсѫждане, разсѫждение, сърдечносѫдовъ, сѫдба, сѫдбина, сѫдбовенъ, сѫдбовно, сѫдбовность, сѫдбоносенъ, сѫдбоносно, сѫдбоносность, сѫдебенъ, сѫдебно, сѫдене, сѫдийски, сѫдийство, сѫдилище, сѫдилищенъ, сѫдимость, сѫдина, сѫдинка, сѫдия, сѫдникъ, сѫдовъ, сѫдодвигателенъ, сѫдокладъ, сѫдопроизводство, сѫдоустройство, сѫдъ (правна институция), сѫдъ (вмѣстилище), сѫдя, сѫждение
	
	\item[Коренъ -СѪК- / -СѪЧ-]: сѫкъ, сѫчка
	
	\item[Коренъ -СѪСТ-]: неприсѫственъ, неприсѫствие, отсѫствамъ, отсѫстване, отсѫстващъ, отсѫственъ, отсѫствие, отсѫствувамъ, отсѫствуване, отсѫствуващъ, присѫствамъ, присѫстване, присѫстващъ, присѫственъ, присѫствие, присѫствувамъ, присѫствуване, присѫствуващъ
	
	\item[Коренъ -СѪЩ-]: вездесѫщность, вездесѫщъ, всѫщность, досѫщъ, единосѫщенъ, единосѫщъ, лесноосѫществимъ, насѫщенъ, насѫщно, насѫщъ, неосѫщественъ, неосѫществимо, неосѫществимость, неосѫществимъ, неприсѫщъ, несѫществено, несѫщественость, несѫщественъ, несѫществуване, несѫществувание, несѫществуващъ, несѫщъ, осѫществимость, осѫществимъ, осѫществитель, осѫществя, осѫществявамъ, осѫществяване, присѫщъ, просѫществувамъ, просѫществуване, самоосѫществяване, спомосѫществуватель, съсѫществувамъ, съсѫществуване, сѫщеврѣмененъ, сѫщеврѣменно, сѫществено, сѫщественость, сѫщественъ, сѫществително, сѫщество, сѫществование, сѫществувамъ, сѫществуване, сѫществувание, сѫществуващъ, сѫщина, сѫщински, сѫщность, сѫщо, сѫщъ
	
	\item[Коренъ -ТРѪБ-]: затрѫбя, затрѫбявамъ, затрѫбяване, протрѫбя, протрѫбявамъ, протрѫбяване, разтрѫбя, разтрѫбявамъ, разтрѫбяване, трѫба, трѫбачъ, трѫбене, трѫбенъ, трѫбестъ, трѫбичка, трѫбовиденъ, трѫбообразенъ, трѫбоподобенъ, трѫбопроводенъ, трѫбопроводъ, трѫбя
	
	\item[Коренъ -ТѪГ- / -ТѪЖ-]: затѫгувамъ, затѫгуване, затѫжа, затѫжавамъ, затѫжаване, затѫженость, натѫжа, натѫжавамъ, натѫжаване, натѫжено, натѫженость, позатѫжа, позатѫжавамъ, позатѫжаване, разтѫжа, разтѫжавамъ, разтѫжаване, разтѫженъ, тѫга, тѫговенъ, тѫговитъ, тѫгувамъ, тѫгуване, тѫжа, тѫжба, тѫженъ, тѫжителка, тѫжитель, тѫжничъкъ, тѫжно, тѫжовенъ, тѫжовно
	
	\item[Коренъ -ТѪД-]: тѫдѣ, тѫдѣва, тѫдѣвашенъ
	
	\item[Коренъ -ТѪП-]: затѫпителенъ, затѫпканъ, затѫпквамъ, затѫпкване, затѫпча, затѫпя, затѫпявамъ, затѫпяване, затѫпѣлость, затѫпѣлъ, затѫпѣя, изпотѫпканъ, изпотѫпквамъ, изпотѫпкване, изпотѫпча, изтѫпанча, изтѫпанчвамъ, изтѫпанчване, изтѫпаня, изтѫпенъ, изтѫпквамъ, изтѫпкване, изтѫпча, натѫпканъ, натѫпквамъ, натѫпкване, натѫпча, неотѫпканъ, неутѫпканъ, отѫпканъ, отѫпквамъ, отѫпкване, отѫпча, отѫпѣя, поразтѫпквамъ, поразтѫпкване, поразтѫпча, потѫпканъ, потѫпквамъ, потѫпкване, потѫпча, притѫпча, притѫпя, притѫпявамъ, притѫпяване, прѣтѫпканость, прѣтѫпканъ, прѣтѫпквамъ, прѣтѫпкване, прѣтѫпча, прѣтѫпя, прѣтѫпявамъ, прѣтѫпяване, разтѫпквамъ, разтѫпкване, разтѫпча, тѫпакъ, тѫпанарка, тѫпанарски, тѫпанарь, тѫпанче, тѫпанъ, тѫпачка, тѫпашки, тѫпащина, тѫпичъкъ, тѫпкало, тѫпкане, тѫпканица, тѫпо, тѫповръхъ, тѫпоглавъ, тѫпость, тѫпота, тѫпотия, тѫпоуменъ, тѫпоумие, тѫпча, тѫпчене, тѫпъ, тѫпя, тѫпѣя, утѫпканъ, утѫпквамъ, утѫпкване, утѫпча
	
	\item[Коренъ -ТѪТ-]: тѫтенъ, тѫтна, тѫтнежъ

	\item[Коренъ -ѪГЬЛ-]: дванадесетоѫгълникъ, десетоѫгъленъ, десетоѫгълникъ, краеѫгъленъ, крайѫгъленъ, многоѫгъленъ, многоѫгълникъ, осмоѫгъленъ, осмоѫгълникъ, остроѫгъленъ, петоѫгъленъ, петоѫгълникъ, правоѫгъленъ, правоѫгълникъ, равноѫгъленъ, седмоѫгъленъ, седмоѫгълникъ, триѫгъленъ, триѫгълникъ, тѫпоѫгъленъ, тѫпоѫгълникъ, четириѫгъленъ, четириѫгълникъ, шестоѫгъленъ, шестоѫгълникъ, ѫглесто, ѫглестъ, ѫгловато, ѫгловатость, ѫгловатъ, ѫгловиденъ, ѫгловъ, ѫглометъръ, ѫгломѣренъ, ѫгломѣръ, ѫглополовяща, ѫгъленъ, ѫгълникъ, ѫгълче, ѫгълъ

 	\end{description}
%%%%%%%%%%

Иванчевскиятъ правописъ прѣмахва дотогавашната употрѣба (т.е. съгласно Дриновския правописъ) на Ѫ и Ѭ въ слѣднитѣ случаи:

\begin{itemize}
	\item Въ пълната и кратка винителна форма на личното мѣстоимение въ 3 л. ед. ч. ж. р.: неѭ, ѭ

	\item Въ 1 л. ед. ч. сег. вр. на глаголитѣ отъ I и II спрежение, напримѣръ: азъ четѫ, азъ пишѫ, азъ играѭ, азъ мълчѫ, азъ молѭ, азъ вървѭ

	\item Въ 3 л. мн. ч. сег. вр. на глаголитѣ отъ всички спрежения, напримѣръ: тѣ четѫтъ, тѣ пишѫтъ, тѣ играѭтъ, тѣ мълчѫтъ, тѣ молѭтъ, тѣ вървѭтъ, тѣ искѫтъ, тѣ излитѫтъ, тѣ отивѫтъ

	\item Въ 3 л. мн. ч. мин. св. вр. и мин. несв. вр. на глаголитѣ, напримѣръ: тѣ четохѫ, тѣ пишехѫ, тѣ играехѫ, тѣ мълчахѫ, тѣ молихѫ, тѣ вървѣхѫ, тѣ искахѫ, тѣ излитахѫ, тѣ отивахѫ
\end{itemize}

 
\end{document}