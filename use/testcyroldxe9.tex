% !TeX spellcheck = bg_BG_IV
% !TeX encoding = UTF-8
% !TeX TS-program = xelatex
\documentclass{article}
\usepackage[X2]{fontenc}
\usepackage[scale=1.0]{OldStandard}

\usepackage[utf8]{inputenc}

%\DeclareUnicodeCharacter{0466}{\CYRLYUS}
%\DeclareTextCommand{\CYRLYUS}{X2}{Я} 
%\DeclareUnicodeCharacter{0467}{\cyrlyus}
%\DeclareTextCommand{\cyrlyus}{X2}{я} 
%\usepackage{ucs}
%\usepackage{autofe}
%\usepackage{newunicodechar}

%\tracinglostchars=3 



% iotated characters
%\newunicodechar{Ѭ}{\CYRII\CYRBYUS}
%\newunicodechar{ѭ}{\cyrii\cyrbyus}
%\newunicodechar{Ꙑ}{\CYRHRDSN\CYRII}
%\newunicodechar{ꙑ}{\cyrhrdsn\cyrii}

%\DeclareTextCommand{\ii}{X2}{ѭ} 


%\newcommand{\II}{Ѭ}
%\newunicodechar{iѫ}{ѭ}

\usepackage{xstring}

\newcommand{\replaceLetter}[3]{%
	\StrSubstitute{#1}{#2}{#3}[\result]%
	\result
}

\renewcommand{\i}[1]{\replaceLetter{#1}{ѫ}{ѭ}}
\newcommand{\I}[1]{\replaceLetter{#1}{Ѫ}{Ѭ}}

%%\catcode`\i=\active
%%\def \i#1{\replaceLetter{#1}{ѫ}{ѭ}}

\usepackage[bulgarian]{babel}

\begin{document}
	\title{Иванчевски правописъ}
	\maketitle

	Речникъ на българския езикъ, съдържащъ слова и тѣхнитѣ форми споредъ изискванията на Иванчевския правописъ. 
	Иванчевски правописъ е името, давано понѣкога на първиятъ официаленъ български правописъ, 
	въвѣденъ съ наредба на министъра на народното просвѣщение Тодоръ Иванчовъ прѣзъ 1899.
	Той се явява поправка на Дриновския правописъ и остава въ сила до въвеждането на Омарчевския правописъ прѣзъ 1921. 
	Иванчевскиятъ правописъ се характеризира съ:
	%%%%%%%%%%
	%
	%%%%%%%%%%
	% Thesaurus checking
	% народъ
	\begin{itemize}
		\item прѣдставяне на звука [ъ] съ буквитѣ \textbf{ъ} и \textbf{ѫ} (голѣма носовка), като различието се прави на етимологическа основа. 
		\item употрѣба на буквата \textbf{ѣ} (ятъ) на етимологическото ѝ мѣсто.
		\item употрѣба на краесловни ерове.
		\item синтактично правило за писане на пъленъ и кратъкъ членъ.
		\item писане на групитѣ ър/ръ и ъл/лъ по североизточното произношение.
		\item писане на глаголнитѣ окончания съ а и я.
	\end{itemize}
	%%%%%%%%%%
	
	Иванчевскиятъ правописъ прѣмахва дотогавашната употрѣба (т.е. съгласно Дриновския правописъ) на Ѫ и \I{Ѭ}   въ слѣднитѣ случаи:
	
	\begin{itemize}
		\item Въ пълната и кратка винителна форма на личното мѣстоимение въ 3 л. ед. ч. ж. р.: не\i{ѫ}, \i{ѫ}
		
		\item Въ 1 л. ед. ч. сег. вр. на глаголитѣ отъ I и II спрежение, напримѣръ: азъ четѫ, азъ пишѫ, азъ играѭ, азъ мълчѫ, азъ молѭ, азъ вървѭ
		
		\item Въ 3 л. мн. ч. сег. вр. на глаголитѣ отъ всички спрежения, напримѣръ: тѣ четѫтъ, тѣ пишѫтъ, тѣ играѭтъ, тѣ мълчѫтъ, тѣ молѭтъ, тѣ вървѭтъ, тѣ искѫтъ, тѣ излитѫтъ, тѣ отивѫтъ
		
		\item Въ 3 л. мн. ч. мин. св. вр. и мин. несв. вр. на глаголитѣ, напримѣръ: тѣ четохѫ, тѣ пишехѫ, тѣ играехѫ, тѣ мълчахѫ, тѣ молихѫ, тѣ вървѣхѫ, тѣ искахѫ, тѣ излитахѫ, тѣ отивахѫ
	\end{itemize}

\end{document}