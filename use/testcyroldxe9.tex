% !TeX spellcheck = bg_BG_IV
% !TeX encoding = UTF-8
% !TeX TS-program = xelatex
\documentclass{article}
\usepackage[X2]{fontenc}
\usepackage[scale=1.0]{OldStandard}

\usepackage[utf8]{inputenc}
\usepackage[bulgarian]{babel}
 
\usepackage{newunicodechar}


%%%%%%%%
% Iotation of Drinov Bulgarian orthography
%%%%%%%%%
\renewcommand{\i}[1]{ѭ}
\newcommand{\I}[1]{Ѭ}
\newunicodechar{і}{\i}
\newunicodechar{І}{\I}
%%%%%%%%%%%%%%%%%%


%%%%%%%%%%
% Document
%%%%%%%%%%
\begin{document}
	\title{Исторически и съврѣменни български правописи}
	\maketitle

	"Иванчевски правописъ" е името, давано понѣкога на първиятъ български държавенъ  правописъ, 
	въвѣденъ съ наредба на министъра на народното просвѣщение Тодоръ Иванчовъ прѣзъ 1899.
	Той се явява прѣправка на прѣдишниятъ Дриновския правописъ и остава въ сила до въвеждането на Омарчевския правописъ прѣзъ 1921. 
	
	Иванчевскиятъ правописъ се характеризира съ:
	%%%%%%%%%%
	%
	%%%%%%%%%%
	% Thesaurus checking
	% народъ
	\begin{itemize}
		\item прѣдставяне на звука [ъ] съ буквитѣ \textbf{ъ}  (ятъ) и \textbf{ѫ} (голѣма носовка), като различието се прави на етимологическа основа. 
		\item употрѣба на буквата \textbf{ѣ}на етимологическото ѝ мѣсто.
		\item изписване на пълния членъ мн. ч. съ наставка -тѣ слѣта с думата.
		\item употрѣба на краесловни ерове.
		\item синтактично правило за писане на пъленъ и кратъкъ членъ.
		\item писане на групитѣ \textit{ър/ръ} и \textit{ъл/лъ} по североизточното произношение.
		\item писане на глаголнитѣ окончания съ \textbf{а} и \textbf{я}, както е обяснено по-долу.
		\item формата \textbf{сѫ} въ 3 л. мн. ч. на глагола \textit{съмъ}.
	\end{itemize}
	%%%%%%%%%%
	
	%%%%%%%%%%%
	%  Sec
	%%%%%%%%%%%%%
	\section{Дриновски правописъ}\label{sec:drinov}
	
		Дриновски-ятъ правописъ залѣга в основата на първия официаленъ български правописъ.  
		Основни-тѣ му положения са дадени по-долу.  
	
		\begin{description} 
		
		\item[Краесловни ерове:] Всички думи завършватъ с графична гласна. 
		Думитѣ, които завършватъ на съгласна при произнасяне, получаватъ краесловенъ -ъ или -ь. С -ь се пишатъ сѫществителни от женски родъ, сѫществителни-тѣ от мѫжки родъ, които получаватъ мека членна форма (день, царь), числителни-тѣ от петь до десеть и производни-тѣ на \textit{десеть}, нѣкои неизмѣняеми думи, производни от горни-тѣ, като сирѣчь и др. 
		С -ь в мѫжки родъ нечленувана форма се пишатъ и прилагателни-тѣ, които в женски родъ окончаватъ на -я: \textit{синь}.	
		
		\item[Падежни окончания:] Изписване на ѫ и іѫ въвъ винителна-та форма и скланяне на члена на сѫществителни-тѣ в ж.р. -- въ горѫ-тѫ, на масѫ-тѫ, прѣдъ кѫщѫ-тѫ и т.н., 
		както и въ пълната и кратка винителна форма на личното мѣстоимение въ 3 л. ед. ч. ж. р.: неіѫ, іѫ.
		
		\item[Употрѣба на членни форми:] при сѫществителни-тѣ от м.р. ед.ч. Дриновъ пише само пъленъ членъ -тъ, добавянъ безъ да се прѣмахва краесловния ер, който по този начинъ се озвучава: ножъ-тъ, царь-тъ, а прилагателни-те в сѫща-та форма членува с -ий. 
		Сѫществителни-тѣ от женски родъ винаги се членуватъ с -та, безъ да се прѣмахва евентуалния краесловенъ ер: жена-та, кость-та. 
		Сѫществителни-тѣ от мн. ч. се членуватъ с -тѣ мѫже-тѣ, жени-тѣ.
		
		\item[Отглаголни съществителни:] се пишѫтъ с окончание -нье: писанье, четенье. С -ь се пишѫт и нѣкои думи, като лозье, здравье и т.н.
	
		\end{description}
	

	
	\begin{itemize}
		\item Въ пълната и кратка винителна форма на личното мѣстоимение въ 3 л. ед. ч. ж. р.: неіѫ, іѫ
		
		\item Въ 1 л. ед. ч. сег. вр. на глаголитѣ отъ I и II спрежение, напримѣръ: 
		нейотирани -- 
			азъ четѫ, азъ пишѫ, азъ мълчѫ, 
		 както и йотиранитѣ --  
		 	азъ играіѫ, азъ моліѫ, азъ вървіѫ;
		 
		\item Въ 3 л. мн. ч. сег. вр. на глаголитѣ отъ всички спрежения, напримѣръ: 
			 нейотирани -- тѣ четѫтъ, тѣ пишѫтъ, тѣ искѫтъ, тѣ излитѫтъ, тѣ мълчѫтъ, тѣ отивѫтъ,
			 както и йотиранитѣ -- 
			тѣ играіѫтъ, тѣ моліѫтъ, тѣ вървіѫтъ;
		
		\item Въ 3 л. мн. ч. мин. св. вр. и мин. несв. вр. на глаголитѣ, напримѣръ: 
			тѣ четохѫ, тѣ пишехѫ, тѣ играехѫ, тѣ мълчахѫ, тѣ молихѫ, тѣ вървѣхѫ, тѣ искахѫ, тѣ излитахѫ, тѣ отивахѫ.
	\end{itemize}

	Иванчевскиятъ правописъ прѣмахва дотогавашната употрѣба (т.е. съгласно Дриновския правописъ) на Ѫ и ІѬ въ горни-тѣ случаи.
	
	%%%%%%%%%%%%%
	% Sec
	%%%%%%%%%%%%%%
	\section{ОФ правопис}

	Това е съвременният български правопис. 
	Характеризира се с произволно въвеждане на я-форми е-форми на мястото на ѣ.
	Елиминира се ѫ като съответните формите се разпадат  на а-форми и ъ-форми.

	%%%%%%%%%%%%%
	% Sec
	%%%%%%%%%%%%%%
	\section{Прѣдимства на старитѣ правописни системи}
	
	За въстановяване на езиковото равновесие като минимумъ трѣбва да се направи слѣдното 
	
	\begin{itemize}
			\item Въстановява се ѣ като обединителна буква. 
			\item Въстановява се ѫ като буква, отбелѣзваща специфичния за българския езикъ тъменъ звукъ.
			\item Въстановява се і като необходимъ за йотацията знакъ.
	\end{itemize}
	В останалитѣ случаи е възможно връщането на етимологичнитѣ форми на ѣ и ѫ като дублетни.
	
	\begin{itemize}
		
	\item При писане случаитѣ с „екане“ и с „якане“ да се отбелѣзватъ винаги съ \textbf{ѣ}. Чрезъ писането на \textbf{ѣ} ще се елиминиратъ случаитѣ на безпринципни изключения, свързани с прилагането на досегашната ОФ норма за писане на е и на я в случаитѣ на промѣнливо \textbf{я}.
	
 	\item При колебание на изговора на дума между а и ъ се пише ѫ.
	
 	\item Въвеждане обратно на буквата ѫ  в глаголнитѣ форми за 1 л. ед. ч. и 3 л. мн. ч. сег. вр., както и въвъ формите на глаголитѣ за тези лица в просто бѫдеще врѣме и за бѫдеще врѣме в миналото.
 	
	\item I се пише въ знацитѣ ІѬ и іѫ, както и при писането на пълната и кратка винителна форма на личното мѣстоимение въ 3 л. ед. ч. ж. р.: неіѫ, іѫ, както и в дателната форма неi, i.
 	
 	\item ІѬ да се пише в глаголнитѣ форми за 1 л. ед. ч. и 3 л. мн. ч. сег. вр., както и въвъ формите на глаголитѣ за тези лица в просто бѫдеще врѣме и бѫдеще врѣме в миналото.
 	
 	
 	
	\end{itemize}

	

\end{document}